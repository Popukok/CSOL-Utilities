\section{进阶教程(供有编程经验的用户参考)}
\label{section_advanced_usage}

\subsection{关于武器攻击}

v1.3.15 之后的版本中,\textbf{常规武器}攻击时将会调用 \lstinline{Weapon.lua} 中的 \lstinline{Weapon.attack} 方法。
该方法采用普通攻击的方式,先按下用户配置好的攻击按钮,然后在接下来的 6 秒内移动视角进行攻击,最后释放攻击按钮。

\begin{minted}[breaklines, breakautoindent]{lua}
---使用武器进行攻击攻击。
---@return nil
function Weapon:attack()
    Mouse:press(self.attack_button)
    local sensitivity_x = 1 - 0.8 * math.random() -- 水平灵敏度∈(0.2, 1]
    local sensitivity_y = 1 - 0.8 * math.random() -- 竖直灵敏度∈(0.2, 1]
    local direction = Utility:random_direction() -- 随机向左或右
    local start_time = DateTime:get_local_timestamp() -- 本次转圈开始时间
    repeat
        local t = Runtime:get_running_time() / 1000
        Mouse:move_relative(math.floor(direction * 100 * sensitivity_x), math.floor(math.sin(t) * 100 * sensitivity_y), Delay.MINI) -- 视角运动:水平方向匀速运动,竖直方向简谐运动
    until (DateTime:get_local_timestamp() - start_time > 6)
    Mouse:release(self.attack_button)
end
\end{minted}

在使用 \lstinline{Weapon.new} 方法创建武器时,您可以列表中重写 \lstinline{attack} 方法。
例如,v1.3.15 提供的 \lstinline{WeaponList.lua} 样例中,就专门为万钧神威重写了 \lstinline{attack} 方法。

\begin{minted}[breakautoindent, breaklines]{lua}
Weapon:new{
    name = "万钧神威",
    switch_delay = Delay.SHORT,
    number = Weapon.MELEE,
    purchase_sequence = {Keyboard.B, Keyboard.NINE, Keyboard.FOUR},
    -- 重写 attack 方法,按照下面定义的方式进行攻击
    attack = function ()
        Mouse:press(Mouse.RIGHT) -- 按下鼠标右键进行范围攻击
        local sensitivity_x = 1 - 0.8 * math.random() -- 水平灵敏度∈(0.2, 1]
        local sensitivity_y = 1 - 0.8 * math.random() -- 竖直灵敏度∈(0.2, 1]
        local direction = Utility:random_direction() -- 随机向左或右
        local start_time = Runtime:get_running_time() -- 本次转圈开始时间
        local first_throw = false
        local second_throw = false
        repeat
            local duration = Runtime:get_running_time() - start_time
            local t = Runtime:get_running_time() / 1000
            Mouse:move_relative(math.floor(direction * 100 * sensitivity_x), math.floor(math.sin(t) * 100 * sensitivity_y), Delay.MINI) -- 视角运动:水平方向匀速,竖直方向简谐
            if (not first_throw and 2000 < duration and duration < 4000)
            then
                Keyboard:click(Keyboard.R, Delay.SHORT)
                first_throw = true
            end
            if (not second_throw and 4000 < duration)
            then
                Keyboard:click(Keyboard.R, Delay.SHORT)
                second_throw = true
            end
        until (Runtime:get_running_time() - start_time > 4500)
        Mouse:release(Mouse.RIGHT) -- 松开鼠标右键释放旋风
    end
}
\end{minted}

对于像圣翼皓印、炽翼魔印、【幽浮】控制核心这种可以于其他武器并行攻击的\textbf{特殊武器},在创建时需要重写 \lstinline{use} 方法。
例如,\lstinline{WeaponList.lua} 样例文件中就为圣翼皓印/炽翼魔印编写了下面的 \lstinline{use} 方法。

\begin{minted}[breakautoindent, breaklines]{lua}
---特殊武器。
---@type Weapon
SpecialWeapon =
    -- 特殊武器圣翼皓印(或炽翼魔印)
    Weapon:new{
        name = "圣翼皓印/炽翼魔印",
        switch_delay = Delay.LONG_LONG,
        number = Weapon.GRENADE ,
        purchase_sequence = {Keyboard.B, Keyboard.EIGHT, Keyboard.NINE},
        discharging = false, -- 是否在释放光印
        discharge_start_moment = 0, --  光印释放的时刻。
        charge_start_moment = 0, -- 充能开始的时刻。
        DISCHARGE_TIME = 25, -- 光印释放时间。
        RECHARGE_TIME = 10, -- 充能时间。
        use = function (self)
            local current_time = DateTime:get_local_timestamp() -- 当前时间戳
            -- 当前正在充能,且充能时间超过 `RECHARGE_TIME`。
            if (not self.discharging and current_time - self.charge_start_moment > self.RECHARGE_TIME)
            then
                self.discharging = true
                self.discharge_start_moment = current_time
                self:switch()
                Mouse:click(Mouse.LEFT, 200)
            -- 当前正在释放,且释放时间超过 `DISCHARGE_TIME`。
            elseif (self.discharging and current_time - self.charge_start_moment > self.DISCHARGE_TIME)
            then
                self.discharging = false
                self.charge_start_moment = current_time
                self:switch()
                -- 按 `Keyboard.R` 召唤界徐圣。 
                Mouse:move_relative(0, 4000, Delay.NORMAL)
                Keyboard:click(Keyboard.R, 350)
                Mouse:move_relative(0, -4000, Delay.NORMAL)
            end
        end
    }\end{minted}

\subsection{关于时间}

\lstinline{DateTime.lua} 中的 \lstinline{DateTime.get_local_timestamp} 用于获取本机 UNIX 时间戳(基于世界协调时,UTC + 00:00,单位为\textbf{秒}),该方法获得的时间戳是否正确取决于 \lstinline{Setting.lua} 文件中配置的时区是否准确。
对时间精度要求不高的情况下,可以使用此函数。

\lstinline{Runtime.lua} 中的 \lstinline{Runtime.get_runnine_time} 用于获取 Lua 模块的运行时间(单位为\textbf{毫秒})。实际使用中,若您需要实现更加精确的定时控制,则可考虑使用本函数。

\subsection{关于中断机制}

罗技软件提供的 Lua 编程接口不支持多线程。
若您学过操作系统原理,则应该清楚多线程实际上需要\textbf{中断机制}来支持。
本项目的 Lua 模块通过封装罗技软件的接口,实现了一种“伪中断”机制。
鉴于发起任何键鼠操作后都需要设定一定的延迟时间(本质上是调用罗技软件提供的 \lstinline{Sleep()} 函数),本项目将 \lstinline{Sleep} 函数封装为 \lstinline{Runtime.sleep()} 方法。
调用此方法时,即认为调用方自发地中断自己的执行,随后在方法内进行中断处理。中断处理完成后,才会执行真正的睡眠,睡眠以 10 毫秒为单位进行,每过 10 毫秒又会进行一次中断处理,这也就实现了文档中提及的“即时中断”功能,
使得用户可以随意地控制 Lua 模块的启停。